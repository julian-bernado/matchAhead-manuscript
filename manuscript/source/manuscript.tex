%
\documentclass[aoas]{imsart}

%% Packages
\RequirePackage{amsthm,amsmath,amsfonts,amssymb}
\RequirePackage[authoryear]{natbib}
\RequirePackage[colorlinks,citecolor=blue,urlcolor=blue]{hyperref}
\RequirePackage{graphicx}

\startlocaldefs
%%%%%%%%%%%%%%%%%%%%%%%%%%%%%%%%%%%%%%%%%%%%%%
%%                                          %%
%% Uncomment next line to change            %%
%% the type of equation numbering           %%
%%                                          %%
%%%%%%%%%%%%%%%%%%%%%%%%%%%%%%%%%%%%%%%%%%%%%%
%\numberwithin{equation}{section}
%%%%%%%%%%%%%%%%%%%%%%%%%%%%%%%%%%%%%%%%%%%%%%
%%                                          %%
%% For Axiom, Claim, Corollary, Hypothesis, %%
%% Lemma, Theorem, Proposition              %%
%% use \theoremstyle{plain}                 %%
%%                                          %%
%%%%%%%%%%%%%%%%%%%%%%%%%%%%%%%%%%%%%%%%%%%%%%
\theoremstyle{plain}
\newtheorem{axiom}{Axiom}
\newtheorem{claim}[axiom]{Claim}
\newtheorem{theorem}{Theorem}[section]
\newtheorem{lemma}[theorem]{Lemma}
%%%%%%%%%%%%%%%%%%%%%%%%%%%%%%%%%%%%%%%%%%%%%%
%%                                          %%
%% For Assumption, Definition, Example,     %%
%% Notation, Property, Remark, Fact         %%
%% use \theoremstyle{definition}            %%
%%                                          %%
%%%%%%%%%%%%%%%%%%%%%%%%%%%%%%%%%%%%%%%%%%%%%%
\theoremstyle{definition}
\newtheorem{definition}[theorem]{Definition}
\newtheorem*{example}{Example}
\newtheorem*{fact}{Fact}
%%%%%%%%%%%%%%%%%%%%%%%%%%%%%%%%%%%%%%%%%%%%%%
%% Please put your definitions here:        %%
%%%%%%%%%%%%%%%%%%%%%%%%%%%%%%%%%%%%%%%%%%%%%%

\endlocaldefs

\begin{document}

\begin{frontmatter}
\title{Efficient Multilevel Matching Using Maximum Flows}
%\title{A sample article title with some additional note\thanksref{t1}}
\runtitle{Efficient Multilevel Matching}

\begin{aug}
%%%%%%%%%%%%%%%%%%%%%%%%%%%%%%%%%%%%%%%%%%%%%%%
%% Only one address is permitted per author. %%
%% Only division, organization and e-mail is %%
%% included in the address.                  %%
%% Additional information can be included in %%
%% the Acknowledgments section if necessary. %%
%% ORCID can be inserted by command:         %%
%% \orcid{0000-0000-0000-0000}               %%
%%%%%%%%%%%%%%%%%%%%%%%%%%%%%%%%%%%%%%%%%%%%%%%
\author[A]{\fnms{Julian}~\snm{Bernado} \ead[label=e1]{bernado@umich.edu}},
\author[A]{\fnms{Katherine}~\snm{Brumberg}}
\and
\author[A]{\fnms{Ben}~\snm{Hansen}}
%%%%%%%%%%%%%%%%%%%%%%%%%%%%%%%%%%%%%%%%%%%%%%
%% Addresses                                %%
%%%%%%%%%%%%%%%%%%%%%%%%%%%%%%%%%%%%%%%%%%%%%%
\address[A]{Department of Statistics,
University of Michigan \printead[presep={ ,\ }]{e1}}

\end{aug}

\begin{abstract}
  Multilevel data structures, where treatment is assigned at the group level but outcomes are measured at the
unit level, pose significant challenges in causal analyses. Traditional matching methods that rely solely on
group-level characteristics often fail to ensure unit-level comparability, leading to biased estimates. Existing
approaches, such as those by Keele and Zubizarreta and Pimentel et al., address this issue by optimizing
unit-level matches for each group pair but face substantial computational burdens when dealing with large
datasets. 

In this paper, we introduce a novel method for multilevel matching that preserves the integrity of unit-level
comparability while significantly reducing computational demands. Our approach utilizes a prognostic model
fitted on historical data to predict unit-level outcomes based on covariates, incorporating both group-level
and mean-centered unit-level covariates. We then compute a simplified yet informative measure of unit-level
comparability for each pair of groups using calipers based on the pairwise contrast standard errors. This
allows us to estimate balance and effective sample size without performing exhaustive optimal matching for
every group pair.
\end{abstract}

\begin{keyword}
\kwd{First keyword}
\kwd{second keyword}
\end{keyword}

\end{frontmatter}
%%%%%%%%%%%%%%%%%%%%%%%%%%%%%%%%%%%%%%%%%%%%%%
%% Please use \tableofcontents for articles %%
%% with 50 pages and more                   %%
%%%%%%%%%%%%%%%%%%%%%%%%%%%%%%%%%%%%%%%%%%%%%%
%\tableofcontents

\section{Introduction}

Multilevel data structures are common in causal analyses: students within schools, patients within hospitals,
households within villages. Often treatment is assigned at the group-level rather than the unit-level in structures of this type for both practical and theoretical reasons. Practically, varying some education intervention
between students in the same school is often infeasible, and from a theoretical standpoint, if units are in
close enough proximity to interact then we lose an often-needed assumption of no-interference. Even if our
outcome is unit-level, such as student test scores or patient diagnostics, we still need to contend with the
hierarchical structure of the data. In matching-based analyses in this setting, we often wish to match groups
that are similar to one another with hopes that the downstream matched sets of units are similar as well.
However, two groups demonstrating closeness on group-level characteristics does not necessarily imply that
there will be a suitable number of comparable unit within these two groups. 

In this paper, I introduce a method for multi-level matching that avoids the naivete of matching only
based on group-level characteristics while also avoiding the computational burden of calculating all optimal
unit-level matches. I do so by considering a simple yet highly informative measure of unit-level comparability
for each pairing of groups. Modern multi-level matching approaches successfully match groups that will have
promising unit-level matches by ”looking ahead” to the best possible matching in each pairing of schools.
However, the work in [4] demonstrates the method's success for a number of schools on the order of one county, raising questions for scalability when analysis is done on all schools in a given state.
Our method
maintains the ”look ahead” ethos of modern strategies, but rather than finding an optimal match in each
pair we take a ”coarser” look at potential unit-matches to save on computation time. I demonstrate the
effectiveness and speed of this method in a case-study using real data from an ongoing collaboration with
the Texas Education Agency. All code is available at the linked GitHub repository, however data is private
and cannot be shared.

\section{Background and problem setting}
Consider a setting with $S$ groups $s = 1, ..., S$ each containing $N_s$ units $u = 1, ..., N_s$ for an overall sample size of $N = \sum_{s=1}^{S} N_s$. Each student has $p$ observed covariates $X_{si} \in \mathbb{R}^{p}$ and treatment indicator $T_{si} \in \{0, 1\}$. We limit our focus to designs where treatment is assigned at the group-level: for all $1 \leq i,j \leq N_s$ we have $T_{si} = T_{sj}$ and hence denote treatment $T_s$. However, we assume we are in the observational setting where the distribution of $T_s$ is unknown. We denote the set of treated and control groups as $S_T$ and $S_C$ respectively with sizes $N_T$ and $N_C$. 

We use the potential outcome framework; each unit has potential outcomes $\{Y_{si}(0), Y_{si}(1)\}$ corresponding to response under control and treatment respectively. We only observe the realized outcome $Y_{si} = Y_{si}(T_{s})$. We consider observational settings analogous to paired clustered randomized control trials where $|S_T|$ pairs of groups are formed and treatment is randomized within each pair. While treatment is assigned at the group-level, we are interested in estimating the unit-level difference in potential outcomes:
\begin{align*}
  \tau &= \mathbf{E}[Y_{si} | T_s = 1 ] - \mathbf{E}[Y_{si} | T_s = 0].
\end{align*}

To estimate $\tau$, we mirror the idealized experiment by forming first-stage matched pairs $M_1, ..., M_{N_T}$ each containing one treatment and one control group. Then, within each group-level matching, we perform an second-stage optimal full matching of treated to control units. We then use [some procedure] to estimate unit-level treatment effect $\tau$. 

The novelty of our method comes from the procedure used to perform the first-stage matching of groups to one another.  A naive approach to matching groups before units might compare group-aggregated covariates $\overline{X}_{s \cdot} \in \mathbb{R}^p$ between treated and control groups and match treated group $s$ to control group $c$ if $\overline{X}_{s \cdot} \approx \overline{X}_{c \cdot}$. However, Zubizarreta shows this can lead to poor matches [cite]. In particular, these matches may be poor because they do not demand closeness at our desired resolution: unit-level comparisons. Recent work in multi-level matching addresses this concern by looking ahead to unit-level matches before deciding first-stage group-level matches. Pimentel et. al introduce an algorithm for producing group-level matches that works as follows:

\begin{enumerate}
  \item Enumerate all pairs $P = \{(t, c) : t\in S_T , c\in S_C\}$ of treated and control groups.
  \item For each pair $(t, c) \in P$, perform an matching of students in $t$ to students in $c$.
  \item Calculate a distance $d(t, c)$ between groups $t$ and $c$ for all pairs $p\in P$ based on the results of the unit-level matching.
  \item Match each group in $S_T$ according to the distance matrix $D$ with $D_{tc} = d(t,c)$.
  \item Perform unit-level matches within each pair of matched groups, then estimate $\tau$ using these matches.
\end{enumerate}

We accept that comparing each treated to control school is a necessary part of creating well-formed matches, but focus on an improvement to the most costly part of their procedure: step 2.


\section{Methodology}
In this section, we discuss the 


\subsection{Practical almost-exact matching}

In the idealized setting, 


% \subsection{Hierarchical modelling and calipers}
% \begin{itemize}
%   \item Introduce basic idea behind hierarchical modelling (unit-level and group-level covariates, averaging and centering, using prior data to predict future data). Mention that propensity scores could be used just as well here.
%   \item Introduce notion of calipers, make reference to common heuristics, then mention that I'm using Ben's calipers
% \end{itemize}

\subsection{Novel matching distances}

\begin{itemize}
  \item Motivate the distances by talking about balance and ess
  \item Introduce the definitions of $e_1$, $e_2$, and $e_3$ and define $d_e(s, c)$ as a function of them.
  \item Mention valuable properties: the ordering, the big O save, the lack of complicated tuning parameters, the fact that it's actually adaptive so as to reduce unnecessary calculation.
\end{itemize}


\section{Comparative analysis using TEA data}

\subsection{TEA data}

\begin{itemize}
  \item motivate problem setting
  \item describe the multi-year multi-grade nature
  \item provide some basic data description
\end{itemize}

\subsection{Experimental setup}

\begin{itemize}
  \item Write up how the prognostic scores and calipers are actually attained in our data
  \item Write up computational experiment procedure
  \item Write up placebo test procedure (incl. throughline matching description)
  \item Describe what Katherine's doing
\end{itemize}

\subsection{Results}

\begin{itemize}
  \item computational comparison results. Plots showing comparative performance of the two: Scatter plot with school size on x-axis, time to complete on y-axis and Pie chart showing how many of our distances stopped after $e_1$, after $e_2$ and $e_3$.
  \item placebo test results. Plot: 12 confidence intervals stacked on one another. One for each grade and subject 3 through 5 and one for each method, colored differently. They all should be centered around zero hopefully. A table showing the bias and variance of each estimate next to one another to see when each method wins. Each method here is: random matching, naive matching, matchAhead matching, then matchMulti matching.
  \item katherine's work results (tbd)
\end{itemize}

\section{Discussion}

\begin{itemize}
  \item tbd, but obviously some summary, some highlighting of what's better about matchAhead distances
\end{itemize}

\section{Questions}
\begin{itemize}
  \item Broadest question is whether this outline makes sense for the whole thing
  \item Too late to add to the method? (for the bias score, it pretty much adds nothing to the comparison for us to look at the average difference between those "matched" by the maxflow calc).
  \item Is this what the authors list will look like or something else?
\end{itemize}


\end{document}
